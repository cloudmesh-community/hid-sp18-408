\section{Jelastic}
Jelastic is a cloud service provider which combines
platform as a service and container as a service in a single
package. 

The main features include buitl-in metering, monitoring 
and troubleshooting tools. It is available as a public, private, 
hybrid and multi-cloud application. It can manage multi tenant 
Docker containerst to native ecosystem. It facilitates live migration
of workloads across various regions and various clouds with 
zero downtime. This makes the system highly reliable during 
migration. All the resources from different cloud environment
can be accessed using a single panel. It also supports 
microservices and legacy application with absolutely no code
changes. It provides integration with Git, SVN and CI/CD tools
and services. It enables scripting to automate processes and events
in the cloud.

In terms of languages, it supports various languages such as Java, PHP, Ruby,
Node.js, Python, .NET and Go. Additionally, it supports virtualization 
technologies like Docker and Virtuozzo. It also supports a wide
range of databases such as MySQL, MariaDB, Percona, PostgreSQL, 
Redis, Neo4j, MongoDB, Cassandra, CouchDB and OrientDB.


\cite{JelasticWiki}\footnote{Please check quotation rules.}\footnote{Please fix labels.}

